\subsection{Paritätische Kommission (PK)}
Die Paritätische Kommission ist eine beratende Sitzung, welche die 
Verwendung von Studiengeldern in einzelnen Projekten diskutiert:
\begin{itemize}
\item Getroffene Beschlüsse gehen als Empfehlungen an den FKR. 
\item Die Besetzung ist paritätisch, also gleich verteilt: zwei Dekanats-Angehörige und zwei Studierende. 
\item Dabei hat jede Person eine Stimme, kann ihr Stimmrecht über-tragen 
oder schriftlich mit einbringen. 
\item Der Erfahrung nach wird im FKR den Empfehlungen der PK gefolgt. 
\item Einige Anträge zur Nutzung von Geldern werden den beiden 
studentischen Mitgliedern meist sofort, manchmal bis spätestens 
einer Woche vorher bekannt gegeben, sodass diese Zeit haben, sich 
auf die Besprechungen vorzubereiten. 
\item Bei unklarer Sachlage oder größeren Anschaffungen werden die 
Antragsteller und/oder Vertreter geladen, um die Situation zu 
schildern.
\end{itemize}