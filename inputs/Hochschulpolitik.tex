\section{Hochschulpolitik}
Die Organisation der Hochschule München besteht aus verschiedenen 
Gremien in denen sich von euch gewählte Vertreter für eure Belange 
einsetzen. Wenn auch ihr euch ehrenamtlich engagieren wollt, meldet 
euch bei der aktiven Fachschaft, wir können immer Hilfe gebrauchen. \doublebreak
Direkt an unserer Fakultät werdet ihr von studentischen Mitgliedern des 
Fakultätsrats, der Paritätischen Kommission und anderen Aus-schüssen 
vertreten. Beispielsweise haben Studierende direkten Einfluss  auf 
Berufungen von Professoren, Verbesserung von Lehrveranstaltungen 
und vieles mehr. \doublebreak
Auch überfakultär setzen sich Studierende für euch ein, z.B. im 
Zentralparitätischen Ausschuss, im Studierendenparlament oder im 
Senat der Hochschule. Sehr bekannte Erfolge dieser Gremien sind 
z.B. die Einführung des Semestertickets oder die Abschaffung  der 
Studiengebühren in Bayern.
\subsection{Hochschulwahlen}
Die Hochschulwahlen finden jährlich, meist im Juni, statt. Ihr wählt eure 
studentischen Vertreter für FKR und StuPa (Studentisches Parlament). 
Zur Wahl aufstellen lassen, kann sich jeder.
\subsection{Paritätische Kommission (PK)}
Die Paritätische Kommission ist eine beratende Sitzung, welche die 
Verwendung von Studiengeldern in einzelnen Projekten diskutiert:
\begin{itemize}
\item Getroffene Beschlüsse gehen als Empfehlungen an den FKR. 
\item Die Besetzung ist paritätisch, also gleich verteilt: zwei Dekanats-Angehörige und zwei Studierende. 
\item Dabei hat jede Person eine Stimme, kann ihr Stimmrecht über-tragen 
oder schriftlich mit einbringen. 
\item Der Erfahrung nach wird im FKR den Empfehlungen der PK gefolgt. 
\item Einige Anträge zur Nutzung von Geldern werden den beiden 
studentischen Mitgliedern meist sofort, manchmal bis spätestens 
einer Woche vorher bekannt gegeben, sodass diese Zeit haben, sich 
auf die Besprechungen vorzubereiten. 
\item Bei unklarer Sachlage oder größeren Anschaffungen werden die 
Antragsteller und/oder Vertreter geladen, um die Situation zu 
schildern.
\end{itemize}
\subsection{Fakultätsrat (FKR)}
Der Fakultätsrat besteht aktuell (Sommersemester 2014) aus 12 
Professoren, 2 sonstigen Mitarbeitern und 4 Studenten und tagt einmal 
im Monat. Eure studentischen Vertreter werden jährlich gewählt und 
treten für eure Rechte ein. \doublebreak
Zu den größten Erfolgen der studentischen Vertreter im Fakultätsrat 
zählt das Verhindern der Abschaffung der Möglichkeit, sich Ausbildungs- und Berufszeit für das Praxissemester anrechnen zu lassen. Solltet ihr also das 
Praxissemester überspringen, denkt daran: Ohne die studentischen 
Vertreter im Fakultätsrat wäre das nicht möglich. Weitere Aufgaben 
des Fakultätsrats umfassen das Abstimmen über finanzielle Anträge 
und Entscheidungen über Änderungen an der Fakultätsordnung. Damit 
wir ein möglichst hohes Stimmgewicht im Fakultätsrat haben, ist eine 
hohe Wahlbeteiligung bei der Hochschulwahl hilfreich. 
\subsection{Studentische Vollversammlung (SVV)}
Jedes Semester organisiert die Fachschaft eine SVV, auf der die 
Studierenden der Fakultät über wichtige Neuerungen jeglicher Art 
informiert werden. Der Dekan stellt sicher, dass jeder Studierende - 
auch du - die Möglichkeit hat, die Veranstaltung zu besuchen. Das heißt, 
Lehrveranstaltungen während der SVV ohne Nachteil für euch 
ausgesetzt.\doublebreak
Ihr solltet hier auf jeden Fall anwesend sein, um die neusten Entscheidungen zu eurem Studium und Details zu
kommenden, wichtigen Veranstaltungen zu erfahren!