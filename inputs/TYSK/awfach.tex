\subsection{AW-Fach-Anmeldung}
In den ersten beiden Semestern kommen zu den Fächern\\ 
eures Studiums noch zwei Allgemein-Wissenschaftliche\\
 Wahlpflichtfächer (kurz AW-Fächer) hinzu. Diese könnt\\
  ihr aus dem Vorlesungsverzeichnis der Fakultät 13 frei wählen.\\
   Ihr findet dieses unter \url{www.gs.hm.edu}. Achtet darauf, dass die ausgesuchten Fächer in euren Stundenplan passen und beachtet auch den Ort der Lehrveranstaltung, so dass ggf. genügend Zeit für einen Campuswechsel gegeben ist. Meldet euch dann unter \url{www.hm.edu/rz/aw-anmeldung} an. Wenn ihr in einem Semester zwei AW-Pflichtfächer belegen möchtet, könnt ihr innerhalb eines Online-Durchgangs automatisch mehrere Prioritätenlisten festlegen. Wenn die Nachfrage größer als das Platzangebot ist, wird die Zuteilung per Losverfahren geregelt. Gebt darum mindestens drei, maximal neun Fächer für jedes gewünschte AW-Fach zur Auswahl in der "Prioritätenliste'' an. Keine Panik, wenn ihr im ersten Losverfahren keinen Platz bekommt. Ihr habt die Möglichkeit, an einem zweiten Losverfahren teilzunehmen. Und selbst danach ist noch eine manuelle Einteilung möglich. Dazu müsst ihr persönlich, mit Studierendenausweis, zur FK13 kommen. Die Ergebnisse der Losverfahren könnt ihr über die Online-Services der Hochschule einsehen. Beide AW-Fächer müssen bis zum Ende des 3. Semesters angetreten werden, da ihr sonst eine Frist 5 kassiert. Schiebt die AW-Fächer also nicht zu lange auf. 