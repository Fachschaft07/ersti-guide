\section{Things You Should Know}

\subsection{AW-Fach-Anmeldung}

In den ersten beiden Semestern kommen zu den Fächern eures Studiums noch zwei Allgemein-Wissenschaftliche Wahlpflichtfächer (kurz AW-Fächer) hinzu. Diese könnt ihr aus dem \textcolor{red}{\textbf{Vorlesungsverzeichnis der Fakultät 13}} frei wählen. Ihr findet dieses unter \url{www.gs.hm.edu}. Achtet darauf, dass die ausgesuchten Fächer in euren Stundenplan passen und beachtet auch den Ort der Lehrveranstaltung, so dass ggf. genügend Zeit für einen Campuswechsel gegeben ist.\doublebreak
Meldet euch unter \url{www.hm.edu/rz/aw-anmeldung} an.\doublebreak
Wenn ihr in einem Semester \textbf{zwei AW-Pflichtfächer} belegen möchtet, könnt ihr innerhalb eines Online-Durchgangs automatisch \textbf{mehrere Prioritätenlisten} festlegen. Wenn die Nachfrage größer als das Platzangebot ist, wird die Zuteilung per Losverfahren geregelt. Gebt darum mindestens drei, maximal neun Fächer für jedes gewünschte AW-Fach zur Auswahl in der "Prioritätenliste'' an. Keine Panik, wenn ihr im ersten Losverfahren keinen Platz bekommt. Ihr habt die Möglichkeit, an einem zweiten Losverfahren teilzunehmen. Und selbst danach ist noch eine manuelle Einteilung möglich. Dazu müsst ihr persönlich, mit Studierendenausweis, zur FK13 kommen.\doublebreak
Die Ergebnisse der Losverfahren könnt ihr über die Online-Services (Primuss) der Hochschule einsehen.\doublebreak
Beide \textcolor{red}{\textbf{AW-Fächer müssen bis zum Ende des 3. Semesters angetreten werden}}, da ihr sonst eine Frist 5 kassiert. Schiebt die AW-Fächer also nicht zu lange auf! 

\subsection{Hochschule- vs. IFW-Account}

Als Studierende an der Hochschule München bekommt ihr bei der \textbf{Immatrikulation} einen Hochschul-Account mit eigener Mail-Adresse. Dazu solltet ihr beim Immatrikulationstag auch einen Infobogen und eine Schritt-für-Schritt-Anleitung bekommen haben.\doublebreak
Mit diesem Hochschul-Account könnt Ihr neben der \textbf{Anmeldung im WLAN und VPN} auch auf das \textbf{Lernportal Moodle} und die zentralen \textbf{Online-Services (Primuss)} der Hochschule München zugreifen.


\subsubsection{Online-Services (Primuss)}
Um an \textcolor{red}{\textbf{Bescheinigungen}} für MVV, BAföG, Immatrikulation oder an euer Notenblatt zu kommen, müsst ihr euch in das Primuss-System der Hochschule einloggen. Dazu benötigt ihr eure Magnetkartennummer auf der Rückseite eures Studierendenausweises. Außerdem wird das System für die Prüfungsanmeldung und die Rückmeldung für das kommende Semester benutzt.\doublebreak
Die Online-Services findet ihr unter \textcolor{red}{\textbf{hm.edu --\textgreater Ich bin Studierender --\textgreater Mein Studium --\textgreater Online-Services --\textgreater PRIMUSS Campus IT}}.

\subsubsection{ifw-Account}

An der Fakultät 07 kommt zum Hochschul-Account noch ein sogenannter ifw-Account hinzu. Diesen braucht ihr, um euch auf den \textcolor{red}{Laborrechnern} und im \textcolor{red}{ZPA-System} anzumelden.\doublebreak
Ihr erhaltet eure ifw-Kennung am ersten Semestertag bei der Führung der Fachschaft oder wenn ihr euch in einem der Computerlabore unter Linux mit dem Benutzer: „register“ und dem Passwort: „FHMFB07“ anmeldet. Hier könnt ihr auch das Passwort für euren ifw-Account ändern. \red{\textbf{Wichtig: Schreibt euch eure ifw-Kennung auf!}}\doublebreak
Habt ihr Probleme mit dem ifw-Account, dann meldet euch bei Herrn Vogler in Raum R1.011.

\subsection{ZPA}

Die Anmeldung zu den Praktika (habt ihr bereits ab dem ersten Semester) und zu den Wahlpflichtfächern (FWP), zum praxisbegleitenden Unterricht, Blockunterricht und Seminaren \textbf{(habt ihr alles in höheren Semestern)} läuft nicht über den Hochschul-Account, sondern wird von der Fakultät selbst im sogenannten ZPA-System vorgenommen. Hier müsst ihr euch mit eurer ifw-Kennung eintragen. Im ersten Semester findet die Anmeldung in den ersten Semesterwochen statt. Ihr solltet auch in der Zukunft auf jeden Fall den Terminplan auf der ZPA-Website berücksichtigen, damit ihr Anmeldungen nicht verpasst:\doublebreak
ZPA-System: \url{w3-o.cs.hm.edu/zpa}\doublebreak
Hinweis: Die Anmeldung zu den Prüfungen und zu den AW-Fächern findet ihr NICHT im ZPA-System sondern in den Online-Services (Primuss) und unter www.hm.edu/rz/aw-anmeldung!

\subsubsection{Studierendenausweis}

Der Studierendenausweis sollte euer ständiger Begleiter werden, da ihr ihn täglich benötigt. Mit validiertem Aufdruck ist der Studierendenausweis auch gleichzeitig euer \textbf{Semesterticket für die öffentlichen Verkehrsmittel} (mehr dazu unter dem Punkt \glqq Semesterticket\grqq{}). In der Mensa und der Cafeteria könnt ihr den Ausweis als \textcolor{red}{Geldkarte} verwenden. Bargeld wird dort nicht akzeptiert. In der Bibliothek könnt ihr mit eurem Studierendenausweis \textcolor{red}{Bücher ausleihen}. Und für die meisten Labore gilt euer Ausweis als \textcolor{red}{Zugangskarte}. Die Kartennummer auf der Rückseite ist gleichzeitig euer \textbf{Passwort für die Online-Services (Primuss)}. Darum gebt diese nicht weiter.

\subsubsection{HM E-Mail}

Der HM-Mail-Account ist sehr wichtig, um auf dem Laufenden zu bleiben. An diese Adresse werden alle, für euer Studium relevanten Informationen und Hinweise, wie Termine für Prüfungsanmeldung usw. gesendet. Achtet darauf, dass ihr regelmäßig eure Mails abruft, oder richtet euch eine Weiterleitung auf eure private E-Mailadresse ein!\doublebreak
Dies geht unter \textcolor{red}{hm.edu --\textgreater Ich bin Studierender --\textgreater Mein Studium --\textgreater Online-Services --\textgreater Zentraler Account}\doublebreak
Mit eurer  \textbf{@hm.edu}  Adresse könnt ihr teilweise auch Angebote für Studierende wie MSDNAA und Matlab nutzen.\doublebreak
Die Hochschule München hat ihr altes Mail System durch Exchange ersetzt. Auf der Seite \url{www.rz.hm.edu/studierende_4/e_mail_}\\ \url{webmail/index.de.html} findet ihr alle Informationen zu diesem Thema.

\subsection{Prüfungen und Noten}
\subsubsection{Prüfungsanmeldung}

Jedes Semester aufs Neue kommt die Zeit, in der ihr euch entscheiden müsst, welche Prüfungen ihr antreten wollt.\doublebreak
Die Prüfungsanmeldung selbst findet über die Online-Services (Primuss) statt. Diese findet ihr auf der Website der Hochschule unter:\doublebreak
\textcolor{red}{hm.edu --\textgreater Ich bin Studierender --\textgreater Mein Studium --\textgreater Online Services  --\textgreater PRIMUSS Campus IT}\doublebreak
Über den Zeitraum der Prüfungsanmeldung werdet ihr über das Schwarze Brett (mehr dazu unter dem Punkt \glqq Schwarzes Brett\grqq{}) und auch mittels einer E-Mail an eure Hochschul-E-Mail-Adresse informiert.\doublebreak
Die Prüfungsanmeldung ist an unserer Hochschule nicht verbindlich. Wenn ihr euch bei einem Fach für eine Prüfung angemeldet habt und nicht zur Prüfung antretet, bekommt ihr, solange keine Fristen anstehen, nicht gleich einen Fünfer. Allerdings müsst ihr euch für jede Prüfung, die ihr schreiben wollt, auch angemeldet haben, da ansonsten die Noten nicht zählen.\doublebreak 
\textcolor{red}{Wichtiger Tipp:} Da es manchmal Probleme mit Prüfungsanmeldungen gibt, solltet ihr immer die Bestätigung der Prüfungsanmeldung ausdrucken oder sicher abspeichern. Denn eine nachträgliche Prüfungs- anmeldung ist zwar durch die Prüfungskommission zwar möglich, eine Anerkennung der Prüfungsleistung allerdings nicht garantiert.

\subsubsection{Schein}
Seit dem Wintersemenster 2014-15 müsst ihr zu Prüfungen, die einen Schein als Prüfungsvoraussetzung haben, diesen auch bei der Prüfung vorlegen können um mitschreiben zu dürfen. Die Möglichkeit eine Prüfung mitzuschreiben und im Anschluss den vergessenen Schein nachzureichen gibt es nicht mehr.

\subsubsection{Notenbekanntgabe und Notenübersicht}

Ein bis zwei Wochen nach der letzten Prüfung ist die Notenbekanntgabe. Ab diesem Zeitpunkt (meist 20 Uhr) könnt ihr eure Noten über die Online-Services (Primuss) anschauen.

\subsubsection{Prüfungseinsicht}

Einen Tag nach der Notenbekanntgabe findet meist die Prüfungs-einsicht statt, bei der ihr zu den Professoren gehen und die geschriebenen Prüfungen noch mal einsehen könnt. Eine nachträgliche Einsicht in die Prüfung ist nicht möglich. Die genauen Zeiten dafür werden von den Professoren über die FI-Services \url{fi.cs.hm.edu} (mehr dazu unter dem Punkt FI-Services) (Prüfungen, Einsichtnahme) bekannt gegeben, oder unter:\doublebreak
\textcolor{red}{\textbf{cs.hm.edu \textgreater Studierende \textgreater Studienorganisation \textgreater Püfungen (Termine, Einsicht, Hilfsmittelkatalog)}} im aktuellen Semester.

\subsubsection{Kann ich bei einem anderen Prof. Prüfungen schreiben?}

Oft scheitert das schon daran, dass Professoren verschiedene Schwerpunkte in ihren Vorlesungen und Klausuren setzen. Das ist eine eher komplexe Frage. Klärt das am besten frühzeitig mit der Prüfungs-kommission ab.

\subsubsection{ECTS-Punkte}

ECTS Punkte sind Punkte, die ihr während eures Studiums sammelt. Die Punkte orientieren sich am Arbeitsaufwand. In der Regel wird auf 25 - 30 Stunden (inkl. Vorlesung und Praktikum) ein ECTS Punkt vergeben. Die für euer Praxissemester und die Bachelorarbeit erforderlichen Punkte findet ihr auf der Hochschulwebsite.\doublebreak
Wenn ihr BAföG bezieht, könnte für euch das Erreichen einer bestimmten ECTS-Punkte-Anzahl ebenfalls wichtig sein. Bitte informiert euch diesbezüglich bei der BAföG Stelle.

\subsection{WLAN und VPN}

In vielen Bereichen der Hochschule steht euch ein Netzzugang über WLAN zur Verfügung. Genauere Beschreibungen zu den folgenden Themen findet ihr hier:

\begin{itemize}
	\item{\url{www.lrz.de/services/netz/mobil/vpn/}}
	\item{\url{www.lrz.de/services/netz/mobil/wireless/}}
\end{itemize}
\textbf{Tipp:} Bitte meldet nicht jedes eurer Geräte im WLAN an, da bei mehreren Geräten pro Person der IP-Adressraum ausgeschöpft wird!

\subsubsection{WLAN: eduroam}

eduroam/DFNRoaming bietet Angehörigen von Mitgliedseinrichtungen einen einfachen Zugang zum Wissenschaftsnetz. Hierbei ist der Zugang sowohl bei der eigenen, als auch bei anderen teilnehmenden wissenschaftlichen Einrichtungen möglich.\doublebreak
Für Angehörige der Hochschule München sieht dies konkret so aus:
\begin{itemize}
		 \item{Äußere Kennung: anonymous@mwn.de\\ (Verschlüsselung: TTLS oder PEAP)}
		 \item{Innere Kennung: \textless username\textgreater @hm.edu\\ (Verschlüsselung: PAP oder MSCHAPv2)}
\end{itemize}


\subsubsection{WLAN: LRZ}

Dieses WLAN ist nicht verschlüsselt, allerdings muss man sich zur Internet-Nutzung des LRZ-Netzes nach der WLAN-Verbindung zu-sätzlich per VPN anmelden.

\subsubsection{VPN}

Eine VPN-Verbindung gestattet den Zugriff auf verschiedene Dienste im Bereich der Hochschule München bzw. im LRZ-Netz. Die Einrichtung von VPN ist daher notwendig, wenn z.B. folgende Dienste genutzt werden wollen:

\begin{itemize}
	\item{Internet-Zugang über das LRZ-WLAN}
	\item{Nutzung öffentlicher LRZ-Netzwerkdosen}
	\item{Nutzung von zugangsbeschränkten, internen Diensten (Beispiele: Remote Update der Sophos Anti-Viren-Software, Zugriff auf beschränkte Bibliotheksdienste wie eBooks oder bestimmte Online- Recherche-Tools)}
\end{itemize}
Zur Nutzung von VPN-Verbindungen ist ein freigeschalteter zentraler Account der Hochschule München und die Installation eines Client-Programms nötig. Dieses findet ihr auf \url{www.lrz.de/services/netz/mobil/vpnclient/}

\subsection{Rückmeldung}

Habt ihr ein Semester hinter euch gebracht, müsst ihr euch unbedingt für das kommende Semester zurückmelden (=Rückmeldung). Darüber wird rechtzeitig auf der Hochschulwebsite informiert unter \textcolor{red}{\textbf{hm.edu \textgreater Ich bin Studierender \textgreater Mein Studium \textgreater Studienverlauf \textgreater\\ Rückmeldung/Lastschrift}}. Außerdem werdet ihr per Mail rechtzeitig darüber in Kenntnis gesetzt. Deswegen: Regelmäßig Mail-Account der Hochschule abrufen!!! Rückmelden könnt ihr euch bis eine Woche nach der Notenbekanntgabe, indem ihr euch entweder für das Lastschriftverfahren eintragt (über den Onlineservic Primuss muss dieses jedes Semester aktiviert werden), oder den Beitrag direkt überweist.\doublebreak
Alle dafür relevanten Informationen, wie z.B. Kontoverbindung der Hochschule etc., findet ihr auf der oben genannten Seite der HM.


\subsection{Semesterticket}

Alle wichtigen Informationen dazu findet ihr unter\\
\url{www.semesterticket-muenchen.de}\doublebreak
Bei der Validierung des Studierendenausweises wird das MVV-Logo aufgedruckt. Dadurch wird der Ausweis zum Sockelticket und ist zusammen mit einem Lichtbildausweis gültig. Der für alle Studierenden verpflichtende Solidarbeitrag beträgt 61€ je Semester und beinhaltet folgende Fahrtberechtigung mit dem \textcolor{red}{\textbf{Studierendenausweis}} mit aufgedrucktem \textcolor{red}{\textbf{MVV-Logo}}:
\begin{itemize}
	\item \textcolor{red}{\textbf{Montag bis Freitag zwischen 18 Uhr und 6 Uhr}} des Folgetages
	\item Samstag, Sonntag, an Feiertagen und am 24.12. und 31.12. ohne
	zeitliche Einschränkungen
\end{itemize}
Mit der IsarCard Semester könnt ihr rund um die Uhr im gesamten Netz der MVG fahren. Die \textcolor{red}{\textbf{IsarCard Semester}} kostet \textcolor{red}{\textbf{152,00}} € und ist eine fakultativ angebotene Zeitkarte für den Zeitraum eines Semesters. Die IsarCard Semester könnt ihr ab September 2015 kaufen. Für den Kauf der IsarCard Semester müsst ihr euren Studierendenausweis mitführen und die Magnetkartennummer des Ausweises, die auf der Rückseite des Ausweises zu finden ist, angeben bzw. eintippen. Bei allen anderen Hochschulen ist die Matrikelnummer anzugeben bzw. einzutippen.

\subsection{Hardware}

\subsubsection{Brauche ich unbedingt einen Laptop zum Studieren?}

An der Hochschule gibt es \textcolor{red}{\textbf{ausreichend Computer-Labore}}, die mit Arbeitsstationen, vor allem für Programmiertätigkeiten, ausgestattet sind. Aber aus ganz persönlicher Erfahrung können wir euch sagen, dass noch keiner am Ende ohne Laptop da stand. Sprich, \textcolor{red}{\textbf{wenn es euer Geldbeutel}} hergibt, empfiehlt es sich einen zuzulegen. Es muss kein teures Gerät sein. Auf besondere Grafikkarten könnt ihr aus Studiensicht verzichten, dafür stehen spezielle Labore zur Verfügung. Ein Kriterium, auf das ihr aber achten solltet, ist die \textcolor{red}{\textbf{Akkulaufzeit}}, da in den Vorlesungssälen nicht genug Steckdosen sind. Ob Linux, Windows oder Mac spielt dabei auch keine besondere Rolle, soweit ihr euch einigermaßen mit eurem Betriebssystem auskennt. Netbooks sind gerade bei den kleinen Tischen in den Vorlesungssälen praktisch, allerdings ist das Programmieren auf zu kleinen Bildschirmen
auf Dauer ziemlich anstrengend. Bei den bekanntesten Entwicklungsumgebungen bleibt neben Toolbars meist nur noch Platz in Größe eines Post-Its.\doublebreak
Zusatz: Bei Anschaffung eines neuen Rechners solltet ihr bedenken, dass in den Vorlesungen nicht auf das von euch verwendete Betriebssystem eingegangen wird. Wählt also eines mit dem ihr gut vertraut seid.\doublebreak
Insofern: \textcolor{red}{\textbf{„Never change a running system!“}}

\subsubsection{Taschenrechner}

Für diverse, vor allem mathematische Klausuren und Prüfungen, werdet ihr einen Taschenrechner benötigen. Laut Prüfungsordnung dürfen diese nicht programmierbar sein. Welchen Taschenrechner, und ob überhaupt einer benutzt werden darf, legt euer Prof. fest. Meistens reichen die Taschenrechner, mit denen ihr im Abitur bzw.
der BOS/FOS-Abschlussprüfung gearbeitet habt, völlig aus. Im Zweifelsfall fragt euren Prof., ob ihr euren Taschenrechner in der Prüfung verwenden dürft.