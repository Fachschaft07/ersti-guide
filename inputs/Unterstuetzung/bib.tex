\subsection{Bibliothek}
Da ihr jetzt Studierende der Hochschule München seid, könnt ihr 
natürlich auch die HM-Bibliothek nutzen :-).
Zum Entleihen der Medien benötigt ihr einen Benutzerausweis. Das ist 
in eurem Fall der Studierenden-Ausweis.
\begin{itemize}
\item Die 11-stellige Nummer auf der Rückseite eures Ausweises ist eure 
Benutzernummer. 
\item Das Kennwort ist voreingestellt als euer Geburtsdatum mit Tag und 
Monat ("TTMM"). Ihr solltet dieses jedoch am besten\\ ändern. 
\end{itemize}
Als Studierende der HM dürft ihr maximal 20 Medien zeitgleich 
ausleihen. Schreibt ihr eure Bachelor- oder Masterarbeit, kann dieses 
Kontingent erhöht werden. Die \red{Leihfristen} für Medien in der Bibliothek
variieren, je nachdem, was ihr ausleiht. Manche Bücher sind zum 
Beispiel mit dem Aufkleber \glqq Semesterleihe\grqq{} gekennzeichnet - sehr 
praktisch für euch, denn so könnt ihr das Buch das ganze Semester 
lang nutzen. Anderenfalls könnt ihr die Medien/Bücher 4 Wochen 
ausleihen. Benötigt ihr das Buch länger, könnt ihr dieses eine Woche 
vor Leihfristende in eurem Konto im OPAC verlängern, solange das 
Buch nicht vorgemerkt wurde. Meldet euch dafür einfach im OPAC \url{www.bib.hm.edu/webopac/}
unter „Mein Konto“ mit der 11-stelligen Benutzernummer und eurem 
Passwort an und klickt auf den grünen Link.\doublebreak
Achtet unbedingt auf die Leihfristen eurer Bücher! Denn gebt ihr die 
Sachen nicht fristgerecht zurück bzw. verlängert sie nicht rechtzeitig, 
bekommt ihr eine \red{Mahnung in Höhe von 7,50 € pro Buch}. Sehr ärgerlich,
gerade, wenn es eh schon stressig ist, mitten im Semester. \doublebreak
Umgehen könnt ihr das ganz einfach, indem ihr regelmäßig euren 
E-Mail-Account überprüft. Die Bibliothek sendet euch ein paar Tage 
vor Leihfristende eine Mail, dass die Bücher zurückgegeben werden 
müssen. Somit habt ihr noch genug Zeit euch darum zu kümmern, 
bzw. schnell das Medium zu verlängern. Allerdings müsst ihr euch dazu 
einmal online einloggen und eure E-Mail Adresse angeben.\doublebreak
Neben der Bibliothek der Hochschule stehen euch natürlich noch 
andere Bibliotheken frei zugänglich zur Verfügung. Die Bayerische 
Staatsbibliothek bietet einen sehr umfangreichen Bestand, zudem 
bekommt ihr da einen Nutzerausweis, mit dem ihr automatisch auch 
noch alle Fach- und Teilbibliotheken sowie die 
Zentralbibliothek der LMU nutzen könnt.