\subsection{Fachschaft}
Die aktive Fachschaft findet ihr ganz hinten links im R-Gebäude im 
Raum \red{R0.013}. Wir verfolgen eure Interessen im Studium und stehen 
euch mit Rat und Tat zur Seite. Dazu bieten wir euch einige Services wie 
Skriptendruck, Getränkeverkauf, Kaff ee und Glühwein im Winter an.\doublebreak
Gerne helfen wir euch bei allen Belangen eures 
Studiums weiter oder wissen an wen ihr euch wenden 
müsst. Und nein, wir werden dafür nicht bezahlt 
und bekommen auch keine anderen Vergütungen. 
Die Arbeit in der Fachschaft ist ehrenamtlich und wir 
freuen uns jederzeit über aktive Mithilfe und Feedback. 
Kommt doch einfach mal vorbei. 

\subsubsection{In der Fachschaft mitmachen}
Wenn ihr Lust habt, bei der aktiven Fachschaft mitzumachen, schreibt
an \email{fachschaftsleitung@fs.cs.hm.edu} eine formlose Mail. Ihr könnt
natürlich auch ohne Mitglied zu werden bei Projekten oder Events
helfen.

\subsubsection{Skriptedruck}
Über die Fachschaft können Skripte kostenlos gedruckt werden. 
Im Moment ist das Drucken nur kursweise möglich. Dazu kann sich 
ein Studierender pro Kurs ein  \red{Skriptendruckformular aus der 
Fachschaft holen} und dieses in seinem Kurs zum Unterschreiben 
herum gehen lassen. Oben auf dem Formular brauchen wir dann nur 
noch die Unterschrift des Professors und die zu druckende Datei und 
wir kümmern uns um den Rest. Die gedruckten Skripte stehen innerhalb von zwei
Wochen in der Fachschaft zur Abholung bereit.

\subsubsection{Skripte- und Prüfungsarchiv}
Wir bieten euch eine Sammlung an alten Prüfungen und Vorlesungsskripten, die wir im Laufe der Jahre gesammelt haben. \doublebreak
Sie stehen online zur Verfügung. Die Zugangsdaten hierzu bleiben jedoch nicht konstant und werden von der Fachschaft bekannt
gegeben. Schaut dazu doch einfach mal vorbei!

\subsubsection{Tutorien}
Gerade in den Praktika ist es sehr wertvoll, Unterstützung von 
erfahrenen Studenten zu bekommen. In vielen Praktika bekommt ihr 
dazu einen studentischen Tutor gestellt, der euch bei der Bearbeitung 
der Praktikumsaufgaben unterstützt. \doublebreak
In höheren Semestern könnt ihr euch als Tutor auch etwas dazu 
verdienen, denn die Professoren suchen immer händeringend nach 
fähigen Tutoren. 

\subsubsection{Stellenanzeigen und Praktika}
Egal ob ihr einen \red{Nebenjob}, eine \red{Praktikumsstelle für Praxissemester}, 
eine \red{Werkstudententätigkeit} oder eine Arbeitsstelle nach dem 
Studium sucht, Angebote gibt es einige. Beispielsweise auf der 
Fachschafts-Website unter \url{http://fs.cs.hm.edu} \arrow \red{INFOS} \arrow \red{Stellenanzeigen} oder im
Jobs-Ordner in der Fachschaft (R0.013). Auch die Fakultät hat einen Job-Ordner im Durchgang vor dem Sekretariat im dritten Stock. Darüber 
hinaus lohnt es sich auch, in die Schaukästen einiger Professoren, die
Stellengesuche von Partnerfirmen aushängen, und die Schaukästen im Gang zur Fachschaft zu schauen.

\subsection{Aufenthaltsräume}
Wer eine Lücke zwischen zwei Vorlesungen überbrücken möchte, sich mit Kommilitonen
zusammen setzen will oder Zeit mit seiner Studienarbeit verbringen darf, findet leider nicht
immer einen freien Vorlesungssaal oder ein Labor. 
Deshalb werden drei Räume für die Studentenschaft bereitgestellt die 
immer offen sind und in der Regel freie Plätze bieten. Steckdosen, LAN-Buchsen und WLAN sind hier ausreichend vorhanden. Die Räume findet 
ihr im EG in der Nähe der Fachschaft (\red{R0.014}, \red{R0.015} \& \red{R0.016})
Bitte sorgt dafür, dass die Aufenthaltsräume gepflegt und vollständig wieder hinterlassen werden.
Leider kam es schon vor, dass Kommilitonen die Räume nicht ordnungsgemäß hinterlassen haben.
Wir wollen die Räume noch länger nutzen und das geht nur, wenn jeder Einzelne mithilft.