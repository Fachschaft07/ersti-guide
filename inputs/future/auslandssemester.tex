\subsection{Auslandssemenster}
"Wer will sich beweisen, der mag die Welt bereisen."\doublebreak
So oder so ähnlich kann man sich das Abenteuer "Auslandssemester" 
vorstellen. Innerhalb der Fakultät 07 hat nur der Studiengang Scientific 
Computing die Pflicht, einen Teil (20 ECTS) im Ausland erfolgreich zu 
absolvieren. In allen anderen Studiengängen ist ein Aufenthalt im 
Ausland fakultativ. \doublebreak
Im Rahmen der Erkundigungen zu einem Auslandssemester, eines
vorweg: Fangt früh an, mindestens ein Jahr vor einem geplanten
Abroad-Trip. Hilfestellungen bietet vor allem International Affairs
mit jeder Menge Partnerhochschulen und Vorträgen. Auch ist die Auslandsbeauftragte der Fakultät noch eine geeignete
Anlaufstation. International Affairs bietet online ebenfalls eine Menge Erfahrungsberichte der bereisten Universitäten.\doublebreak
Ansonsten sei gesagt, dass ein Aufenthalt im Ausland einen auf 
unterschiedliche Weise herausfordert. Innerhalb Europas oder anderer 
"westlicher" Länder befindet man sich zumeist in kleineren kulturellen
und sprachlichen Herausforderungen. \doublebreak
Sie bieten einen guten Einstieg in andere Welten und eine Bildung 
erster interkultureller Kompetenzen. Ein Trip zu anderen Kontinenten,
verbunden mit komplett anderen Kulturen wie der fernöstlichen Welt, 
fordert einen darüber hinaus mehr, gibt aber auch mehr zurück.\doublebreak 
Zusammenfassend sollte man die meist einmalige Chance eines 
Auslandsaufenthaltes versuchen zu nutzen, rechtzeitig planen und 
ordentlich durchziehen und wenn es das Erlernen einer anderen 
Sprache ist! 